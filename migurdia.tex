% !TEX TS-program = xelatex
% !TEX encoding = UTF-8 Unicode
% !Mode:: "TeX:UTF-8"

% migurdia: semantic-based LaTeX resume template
% Copyright (C) 2021 jeffswt
% -----------------------------------------------------------------------------
% Permission is hereby granted, free of charge, to any person obtaining a copy
% of this software and associated documentation files (the “Software”), to
% deal in the Software without restriction, including without limitation the
% rights to use, copy, modify, merge, publish, distribute, sublicense, and/or
% sell copies of the Software, and to permit persons to whom the Software is
% furnished to do so, subject to the following conditions:
%
% The above copyright notice and this permission notice shall be included in
% all copies or substantial portions of the Software.
%
% THE SOFTWARE IS PROVIDED “AS IS”, WITHOUT WARRANTY OF ANY KIND, EXPRESS OR
% IMPLIED, INCLUDING BUT NOT LIMITED TO THE WARRANTIES OF MERCHANTABILITY,
% FITNESS FOR A PARTICULAR PURPOSE AND NONINFRINGEMENT. IN NO EVENT SHALL THE
% AUTHORS OR COPYRIGHT HOLDERS BE LIABLE FOR ANY CLAIM, DAMAGES OR OTHER
% LIABILITY, WHETHER IN AN ACTION OF CONTRACT, TORT OR OTHERWISE, ARISING
% FROM, OUT OF OR IN CONNECTION WITH THE SOFTWARE OR THE USE OR OTHER DEALINGS
% IN THE SOFTWARE.
% -----------------------------------------------------------------------------
% Migurdia aims at creating a resume template whose body is entirely free of
% formatting and styling except those specifically specified by the user. All
% sections and data groups are customized through key-value arguments and can
% be used or interpreted with ease.
%
% The document elements are categorized into 3 classes:
%  1. sections - the largest classification of all
%  2. clusters - each event or experience consists one single cluster
%  3. items - specific aspects of clusters, or sections if no clusters apply
%
% 4 prebuilt sections are provided, namely the following:
%  1. skills - fundamental programming, technical skills or abilities
%  2. awards - prizes or honours received
%  3. experiences - primarily work experiences, mostly of internships
%  4. projects - small projects conducted while being sleepless
%
% To project the usage of Migurdia more vividly, an example is provided
% directly as follows:
% -----------------------------------------------------------------------------
% % !TEX TS-program = xelatex
% % !TEX encoding = UTF-8 Unicode
% % !Mode:: "TeX:UTF-8"
%
% % !TEX TS-program = xelatex
% !TEX encoding = UTF-8 Unicode
% !Mode:: "TeX:UTF-8"

% migurdia: semantic-based LaTeX resume template
% Copyright (C) 2021-2024 jeffswt
% -----------------------------------------------------------------------------
% Permission is hereby granted, free of charge, to any person obtaining a copy
% of this software and associated documentation files (the “Software”), to
% deal in the Software without restriction, including without limitation the
% rights to use, copy, modify, merge, publish, distribute, sublicense, and/or
% sell copies of the Software, and to permit persons to whom the Software is
% furnished to do so, subject to the following conditions:
%
% The above copyright notice and this permission notice shall be included in
% all copies or substantial portions of the Software.
%
% THE SOFTWARE IS PROVIDED “AS IS”, WITHOUT WARRANTY OF ANY KIND, EXPRESS OR
% IMPLIED, INCLUDING BUT NOT LIMITED TO THE WARRANTIES OF MERCHANTABILITY,
% FITNESS FOR A PARTICULAR PURPOSE AND NONINFRINGEMENT. IN NO EVENT SHALL THE
% AUTHORS OR COPYRIGHT HOLDERS BE LIABLE FOR ANY CLAIM, DAMAGES OR OTHER
% LIABILITY, WHETHER IN AN ACTION OF CONTRACT, TORT OR OTHERWISE, ARISING
% FROM, OUT OF OR IN CONNECTION WITH THE SOFTWARE OR THE USE OR OTHER DEALINGS
% IN THE SOFTWARE.
% -----------------------------------------------------------------------------
% Migurdia aims at creating a resume template where the body is mostly free of
% formatting or styling except those inevitably required by the user. All
% sections and data groups are customized through key-value arguments and can
% be used or interpreted with ease.
%
% The document elements are organized into 3 levels:
%  1. sections - the largest classification of all
%  2. entries - each single thing consists one entry
%  3. items - equivalent to the bullet points in a list
%
% 4 prebuilt sections are provided, namely the following:
%  1. skills - fundamental programming, technical skills or abilities, or
%     alternatively, 'brief'
%  2. awards - prizes or honours received
%  3. experiences - primarily work experiences, mostly of internships
%  4. projects - small projects conducted while being sleepless
%
% You may acquire a template right aside Migurdia.

\documentclass[a4paper,11pt]{article}

% -------------------------------- packages -----------------------------------

\usepackage[empty]{fullpage}  % page size limitations
\usepackage{titlesec}  % section format definition
\usepackage{enumitem}  % units of measurement (?)
\usepackage[hidelinks]{hyperref}  % hyperlinks
\usepackage{keycommand}  % key-value command definitions
\usepackage{ragged2e}  % used to justify itemize-s
\usepackage{xeCJK}  % allow cjk characters in xelatex
\usepackage{fontspec}  % allow text formatting
\usepackage[symbol]{footmisc}  % fancy footnotes

% ------------------------------ font config ----------------------------------

% * uncomment these if you require CJK character support and have those fonts
%   installed in your fonts library
% * run `fc-cache -f` before xelatex if you experience lags

% \newcommand{\resumeFontSerif}{Adobe Song Std}
% \newcommand{\resumeFontSans}{Adobe Heiti Std}
% \newcommand{\resumeFontMono}{Adobe Fangsong Std}
% \newcommand{\resumeFontScript}{Adobe Kaiti Std}
% \newcommand{\resumeFontLishu}{Lisu}
% \newcommand{\resumeFontYouyuan}{YouYuan}

% \setCJKmainfont[
%     BoldFont = \resumeFontSans,
%     ItalicFont = \resumeFontScript,
%     SmallCapsFont = \resumeFontSans,
% ]{\resumeFontSerif}
% \setCJKsansfont{\resumeFontSans}
% \setCJKmonofont{\resumeFontMono}

% ----------------------------- style config ----------------------------------

% custom padding sizes, scaled to font size
\newcommand{\resumeStyleSectionPaddingBefore}{\vspace{-0.6em}}
\newcommand{\resumeStyleClusterPaddingBefore}{\vspace{-0.68em}}
\newcommand{\resumeStyleItemListPaddingBefore}{\vspace{-0.8em}}
\newcommand{\resumeStyleItemPaddingBefore}{\vspace{-0.32em}}
\newcommand{\resumeStyleSectionPaddingAfter}{\vspace{-0.6em}}

% hotfix for style defects, scaled to font size
\newcommand{\resumeStylefixSkillItem}{\vspace{-0.28em}}
\newcommand{\resumeStylefixEducationSection}{\vspace{-0.1em}}
\newcommand{\resumeStylefixAwardItem}{\vspace{-1.36em}}
\newcommand{\resumeStylefixAwardSectionBefore}{\vspace{1.2em}}
\newcommand{\resumeStylefixAwardSectionAfter}{\vspace{-0.84em}}
\newcommand{\resumeStylefixSingleSubheading}{\vspace{0.6em}}

% adjust page margins
\addtolength{\oddsidemargin}{-0.5in}
\addtolength{\evensidemargin}{-0.5in}
\addtolength{\textwidth}{1in}
\addtolength{\topmargin}{-.5in}
\addtolength{\textheight}{1.0in}

% justified alignment
\raggedbottom
\raggedright

% tabular column separator lengths
\setlength{\tabcolsep}{0in}

% section formatting
\titleformat{\section}{
  \resumeStyleSectionPaddingBefore\scshape\raggedright\large
}{}{0em}{}[\titlerule]

% url style stays the same
\urlstyle{same}

% -------------------------------- functions ----------------------------------

% @brief \labelitemii -- renewed bullet icon
\renewcommand{\labelitemii}{
    $\vcenter{\hbox{\tiny$\bullet$}}$
}

% @brief \resumeHeading -- header containing personal information, must be
%        used before anything appears in the document
% @param name: resume author's name
% @param mobile: mobile phone in format (+..) ...-....-.... or similar
% @param email: user email, creates hyperlink also
% @param homepage: user website, do not include http/https schema
\newkeycommand{\resumeHeading}[
        name = John Smith,
        mobile = (+86) 138-3838-5438,
        email = example@example.com,
        homepage = example.com,
]{
    \newpage \textbf{\Huge \commandkey{name}}
    \\ \vspace{0.1em} \small \commandkey{mobile}
    $|$ \href{mailto:\commandkey{email}}{\underline{\commandkey{email}}}
    $|$ \href{https://\commandkey{homepage}}{\underline{\commandkey{homepage}}}
    \vspace{0.8em}
}

% @brief \begin{resumeSection} -- contains a resume section within, creating
%        title and section decoration along
% @param 1: title of this section
% @note: if padding appears to behave strangely, modify the relevant padding
%        values in the project header
\newenvironment{resumeSection}[1]{
    \resumeStyleSectionPaddingBefore
    \section{#1}
    \begin{itemize}[leftmargin=0.007\textwidth, label={}]
}{
    \end{itemize}
    \resumeStyleSectionPaddingAfter
}

% @brief \begin{resumeSkillSection} -- technical skills section
\newenvironment{resumeSkillSection}{
    \begin{resumeSection}{Technical Skills}
}{
    \end{resumeSection}
}

% @brief \begin{resumeBriefSection} -- brief information
\newenvironment{resumeBriefSection}{
    \begin{resumeSection}{Brief}
}{
    \end{resumeSection}
}

% @brief \begin{resumeEducationSection} -- education section
% @note: if padding appears to behave strangely, change the hotfix value
\newenvironment{resumeEducationSection}{
    \begin{resumeSection}{Education}
}{
    \end{resumeSection}
    \resumeStylefixEducationSection
}

% @brief \begin{resumeAwardSection} -- awards section
% @note: if padding appears to behave strangely, change the hotfix value
\newenvironment{resumeAwardSection}{
    \begin{resumeSection}{Awards}
        \resumeStylefixAwardSectionBefore
}{
    \end{resumeSection}
    \resumeStylefixAwardSectionAfter
}

% @brief \begin{resumeExperienceSection} -- experience section
\newenvironment{resumeExperienceSection}{
    \begin{resumeSection}{Experience}
}{
    \end{resumeSection}
}

% @brief \begin{resumeProjectSection} -- projects section
\newenvironment{resumeProjectSection}{
    \begin{resumeSection}{Projects}
}{
    \end{resumeSection}
}

% @brief \resumeSubheadingTwoLinesInternal -- internal function showing
%        subheading composed of 4 units or 2 lines, in order:
%        **Title**                                               Remark
%        *Subtitle line*                                     Sub-remark
% @param 1: Title to be used with bold text, lies on the upper-left
% @param 2: Subtitle to be used in small italic, lies on the lower-left
% @param 3: Remark in regular style, lies on the upper-right
% @param 4: Sub-remark in small italic, lies on the lower-right
\newcommand{\resumeSubheadingTwoLinesInternal}[4]{
    \resumeStyleClusterPaddingBefore \item
    \begin{tabular*}{0.986\textwidth}[t]{l@{\extracolsep{\fill}}r}
        \textbf{#1} & #3 \\
        \textit{\small #2} & \textit{\small #4} \\
    \end{tabular*}
}

% @brief \resumeSubheadingOneLineInternal -- internal function showing
%        subheading composed of 2 units or 1 line, in order:
%        **Title**                                               Remark
% @param 1: Title to be used with bold text, lies on the upper-left
% @param 2: Remark in regular style, lies on the upper-right
% @note: if padding appears to behave strangely, change the hotfix value
\newcommand{\resumeSubheadingOneLineInternal}[2]{
    \resumeStyleClusterPaddingBefore \item
    \begin{tabular*}{0.986\textwidth}{l@{\extracolsep{\fill}}r}
        \small #1 & #2 \\
    \end{tabular*}
    \resumeStylefixSingleSubheading
}

% @brief \resumeEntryItemList -- internal function used to contain items
% @param 1: a list of \resumeItem (s)
\newcommand{\resumeEntryItemList}[1]{
    \resumeStyleItemListPaddingBefore
    \begin{itemize}[leftmargin=1.6em, rightmargin=0.007\textwidth]
        \justifying
        #1
    \end{itemize}
}

% @brief \resumeSkillEntry -- entry of a single (group of) skill
% @param title: whichever aspect that is described, e.g. Languages, Tools
% @param list: list of skills, in plain text
\newkeycommand{\resumeSkillEntry}[
        title = Skill,
        list = List skills here,
]{
    \resumeStylefixSkillItem \resumeStyleItemPaddingBefore
    \small \item \textbf{\commandkey{title}}{: \commandkey{list}}
}

% @brief \resumeEducationEntry -- describing education received
% @param institute: which university / high school / etc. did you attend
% @param role: which major / minor were you under pursuit
% @param beginTime: when did this education begin
% @param endTime: when did / will this education end (Present if ongoing)
% @param location: describe location (city / country / region)
\newkeycommand{\resumeEducationEntry}[
        institute = University of Claydon,
        role = Undergraduate student in Computer Science,
        beginTime = Jan. 1970,
        endTime = Present,
        location = {Claydon, Atlantica},
]{
    \resumeSubheadingTwoLinesInternal
        {\commandkey{institute}}
        {\commandkey{role}}
        {\commandkey{beginTime} -- \commandkey{endTime}}
        {\commandkey{location}}
}

% @brief \resumeAwardEntry -- describing award received
% @param prize: level of award, e.g. Silver medal / First prize / etc.
% @param name: name of award or canonical competition name
% @param time: time of competition taken place (or award received otherwise)
\newkeycommand{\resumeAwardEntry}[
        prize = Metal Medal,
        name = ICPC 1970 Claydon Regionals,
        time = Jan. 1970,
]{
    \resumeStylefixAwardItem \resumeStyleItemPaddingBefore \item
    % this part share the same code as \resumeSubheadingOneLineInternal does,
    % but does not use due to padding semantics' reasons.
    \begin{tabular*}{0.986\textwidth}{l@{\extracolsep{\fill}}r}
        \small \textit{\commandkey{prize}}, {\commandkey{name}} &
            \commandkey{time} \\
    \end{tabular*}
}

% @brief \resumeExperienceEntry -- enumerating work experience
% @param role: work role during participation
% @param institute: company or other institute which you worked at
% @param beginTime: begin time of this work experience
% @param endTime: end time of this work experience (Present if ongoing)
% @param location: location of the institute (Remote if wfh)
% @param 1: list of cluster items
\newkeycommand{\resumeExperienceEntry}[
        role = Internship in Software Engineering,
        institute = Claydon Corporation,
        beginTime = Jan. 1970,
        endTime = Present,
        location = {Claydon, Atlantica},
][1]{
    \resumeSubheadingTwoLinesInternal
        {\commandkey{role}}
        {\commandkey{institute}}
        {\commandkey{beginTime} -- \commandkey{endTime}}
        {\commandkey{location}}
    \resumeEntryItemList{#1}
}

% @brief \resumeProjectEntry -- describing a project involved
% @param name: human-friendly name of project
% @param framework: languages, tech stacks or frameworks involved
% @param duration: begin -- termination time of development
% @param 1: list of cluster items
\newkeycommand{\resumeProjectEntry}[
        name = Example Project,
        frameworks = Assembly,
        duration = {Jan. 1970 -- Present},
][1]{
    \resumeSubheadingOneLineInternal
        {\textbf{\commandkey{name}} $|$ \emph{\commandkey{frameworks}}}
        {\commandkey{duration}}
    \resumeEntryItemList{#1}
}

% @brief \resumeItem -- entry item, preceding plain text but do not require
%        any braces around the plain text following it
\newcommand{\resumeItem}{
    \resumeStyleItemPaddingBefore \item \small
}

% @brief \resumeWebLink -- lazily create a clickable hyperlink
% @param 1: link with https:// schema prepended
\newcommand{\resumeWebLink}[1]{\href{#1}{\underline{#1}}}

%
% \begin{document}
%
% \resumeHeading[
%     name = John Smith,
%     mobile = (+86) 138-3838-5438,
%     email = example@example.com,
%     homepage = example.com,
% ]
%
% \begin{resumeSkillSection}
% \resumeSkillCluster[
%     aspect = Languages,
%     enlist = {Java, Rust, Go, OCaml},
% ]
% \resumeSkillCluster[
%     aspect = Frameworks and Tools,
%     enlist = {PyTorch, Tensorflow, mxnet, Keras},
% ]
% \end{resumeSkillSection}
%
% \begin{resumeEducationSection}
% \resumeEducationCluster[
%     institute = University of Claydon,
%     role = Undergraduate student in Computer Science,
%     beginTime = Jan. 2038,
%     endTime = Present,
%     city = Claydon,
%     region = Atlantica,
% ]
% \resumeEducationCluster[
%     institute = HS Affiliated to University of Claydon,
%     role = High school student,
%     beginTime = Jan. 1970,
%     endTime = Jan. 2038,
%     city = Claydon,
%     region = Atlantica,
% ]
% \end{resumeEducationSection}
%
% \begin{resumeAwardSection}
% \resumeAwardCluster[
%     prize = Metal medal,
%     name = 44th ICPC Atlantica Regional Contest Claydon Site,
%     time = Jan. 2038,
% ]
% \resumeAwardCluster[
%     prize = Metal medal,
%     name = 1970 Atlantica Collegiate Programming Contest,
%     time = Jan. 1970,
% ]
% \end{resumeAwardSection}
%
% \begin{resumeExperienceSection}
% \resumeExperienceCluster[
%     role = Internship in Software Engineering,
%     institute = Claydon Corporation,
%     beginTime = Jan. 1970,
%     endTime = Jan. 2038,
%     location = {Claydon, Atlantica},
% ]{
%     \resumeItem Dropped all databases from a production database.
%     \resumeItem Restored the production database from scratch by calculating
%         residuals inferred from orbiting electrons.
% }
% \resumeExperienceCluster[
%     role = Professor in Quantum Physics,
%     institute = University of Claydon,
%     beginTime = Jan. 2038,
%     location = {Claydon, Atlantica},
% ]{
%     \resumeItem Held lectures on how to drop production databases.
%     \resumeItem Held lectures on restoring production database from scratch.
% }
% \end{resumeExperienceSection}
%
% \begin{resumeProjectSection}
% \resumeProjectCluster[
%     name = Automatic Database Dropper,
%     frameworks = {Shell},
%     beginTime = Jan. 1970,
%     endTime = Jan. 2038,
% ]{
%     \resumeItem Link to project: \resumeWebLink{https://example.com}
%     \resumeItem Drops database on script download.
% }
% \resumeProjectCluster[
%     name = Database Recovery Utility,
%     frameworks = {SQL},
%     beginTime = Jan. 2038,
%     endTime = Present,
% ]{
%     \resumeItem Link to project: \resumeWebLink{https://example.com}
%     \resumeItem Restores databases out of nowhere.
%     \resumeItem Does \textbf{not} guarantee data intactness.
% }
% \end{resumeProjectSection}
%
% \end{document}
% -----------------------------------------------------------------------------

\documentclass[a4paper,11pt]{article}

% -------------------------------- packages -----------------------------------

\usepackage[empty]{fullpage}  % page size limitations
\usepackage{titlesec}  % section format definition
\usepackage{enumitem}  % units of measurement (?)
\usepackage[hidelinks]{hyperref}  % hyperlinks
\usepackage{keycommand}  % key-value command definitions
\usepackage{ragged2e}  % used to justify itemize-s
\usepackage{xeCJK}  % allow cjk characters in xelatex

% ------------------------------ font config ----------------------------------

% * uncomment these if you require CJK character support and have those fonts
%   installed in your fonts library
% * run `fc-cache -f` before xelatex if you experience lags

% \newcommand{\resumeFontSerif}{Adobe Song Std}
% \newcommand{\resumeFontSans}{Adobe Heiti Std}
% \newcommand{\resumeFontMono}{Adobe Fangsong Std}
% \newcommand{\resumeFontScript}{Adobe Kaiti Std}
% \newcommand{\resumeFontLishu}{Lisu}
% \newcommand{\resumeFontYouyuan}{YouYuan}

% \setCJKmainfont[
%     BoldFont = \resumeFontSans,
%     ItalicFont = \resumeFontScript,
%     SmallCapsFont = \resumeFontSans,
% ]{\resumeFontSerif}
% \setCJKsansfont{\resumeFontSans}
% \setCJKmonofont{\resumeFontMono}

% ----------------------------- style config ----------------------------------

% custom padding sizes, scaled to font size
\newcommand{\resumeStyleSectionPaddingBefore}{\vspace{-0.6em}}
\newcommand{\resumeStyleClusterPaddingBefore}{\vspace{-0.68em}}
\newcommand{\resumeStyleItemListPaddingBefore}{\vspace{-0.8em}}
\newcommand{\resumeStyleItemPaddingBefore}{\vspace{-0.32em}}
\newcommand{\resumeStyleSectionPaddingAfter}{\vspace{-0.6em}}

% hotfixes for style defects, scaled to font size
\newcommand{\resumeStylefixSkillItem}{\vspace{-0.28em}}
\newcommand{\resumeStylefixEducationSection}{\vspace{-0.1em}}
\newcommand{\resumeStylefixAwardItem}{\vspace{-1.36em}}
\newcommand{\resumeStylefixAwardSectionBefore}{\vspace{1.2em}}
\newcommand{\resumeStylefixAwardSectionAfter}{\vspace{-0.84em}}
\newcommand{\resumeStylefixSingleSubheading}{\vspace{0.6em}}

% adjust page margins
\addtolength{\oddsidemargin}{-0.5in}
\addtolength{\evensidemargin}{-0.5in}
\addtolength{\textwidth}{1in}
\addtolength{\topmargin}{-.5in}
\addtolength{\textheight}{1.0in}

% justifyed alignment
\raggedbottom
\raggedright

% tabular column separator lengths
\setlength{\tabcolsep}{0in}

% section formatting
\titleformat{\section}{
  \resumeStyleSectionPaddingBefore\scshape\raggedright\large
}{}{0em}{}[\titlerule]

% url style stays the same
\urlstyle{same}

% -------------------------------- functions ----------------------------------

% @brief \labelitemii -- renewed bullet icon
\renewcommand{\labelitemii}{
    $\vcenter{\hbox{\tiny$\bullet$}}$
}

% @brief \resumeHeading -- header containing personal information, must be
%        used before anything appears in the document
% @param name: resume author's name
% @param mobile: mobile phone in format (+..) ...-....-.... or similar
% @param email: user email, creates hyperlink also
% @param homepage: user website, do not include http/https schema
\newkeycommand{\resumeHeading}[
        name = John Smith,
        mobile = (+86) 138-3838-5438,
        email = example@example.com,
        homepage = example.com,
]{
    \newpage \textbf{\Huge \commandkey{name}}
    \\ \vspace{0.1em} \small \commandkey{mobile}
    $|$ \href{mailto:\commandkey{email}}{\underline{\commandkey{email}}}
    $|$ \href{https://\commandkey{homepage}}{\underline{\commandkey{homepage}}}
    \vspace{0.8em}
}

% @brief \begin{resumeSection} -- contains a resume section within, creating
%        title and section decoration alongs
% @param 1: title of this section
% @note: if padding appears to behave strangely, modify the relevant padding
%        values in the project header
\newenvironment{resumeSection}[1]{
    \resumeStyleSectionPaddingBefore
    \section{#1}
    \begin{itemize}[leftmargin=0.007\textwidth, label={}]
}{
    \end{itemize}
    \resumeStyleSectionPaddingAfter
}

% @brief \begin{resumeSkillSection} -- technical skills section
\newenvironment{resumeSkillSection}{
    \begin{resumeSection}{Technical Skills}
}{
    \end{resumeSection}
}

% @brief \begin{resumeEducationSection} -- education section
% @note: if padding appears to behave strangely, change the hotfix value
\newenvironment{resumeEducationSection}{
    \begin{resumeSection}{Education}
}{
    \end{resumeSection}
    \resumeStylefixEducationSection
}

% @brief \begin{resumeAwardSection} -- awards section
% @note: if padding appears to behave strangely, change the hotfix value
\newenvironment{resumeAwardSection}{
    \begin{resumeSection}{Awards}
        \resumeStylefixAwardSectionBefore
}{
    \end{resumeSection}
    \resumeStylefixAwardSectionAfter
}

% @brief \begin{resumeExperienceSection} -- experience section
\newenvironment{resumeExperienceSection}{
    \begin{resumeSection}{Experience}
}{
    \end{resumeSection}
}

% @brief \begin{resumeProjectSection} -- projects section
\newenvironment{resumeProjectSection}{
    \begin{resumeSection}{Projects}
}{
    \end{resumeSection}
}

% @brief \resumeSubheadingTwoLinesInternal -- internal function showing
%        subheading composed of 4 units or 2 lines, in order:
%        **Title**                                               Remark
%        *Subtitle line*                                      Subremark
% @param 1: Title to be used with bold text, lies on the upper-left
% @param 2: Subtitle to be used in small italic, lies on the lower-left
% @param 3: Remark in regular style, lies on the upper-right
% @param 4: Subremark in small italic, lies on the lower-right
\newcommand{\resumeSubheadingTwoLinesInternal}[4]{
    \resumeStyleClusterPaddingBefore \item
    \begin{tabular*}{0.986\textwidth}[t]{l@{\extracolsep{\fill}}r}
        \textbf{#1} & #3 \\
        \textit{\small #2} & \textit{\small #4} \\
    \end{tabular*}
}

% @brief \resumeSubheadingOneLineInternal -- internal function showing
%        subheading composed of 2 units or 1 line, in order:
%        **Title**                                               Remark
% @param 1: Title to be used with bold text, lies on the upper-left
% @param 2: Remark in regular style, lies on the upper-right
% @note: if padding appears to behave strangely, change the hotfix value
\newcommand{\resumeSubheadingOneLineInternal}[2]{
    \resumeStyleClusterPaddingBefore \item
    \begin{tabular*}{0.986\textwidth}{l@{\extracolsep{\fill}}r}
        \small #1 & #2 \\
    \end{tabular*}
    \resumeStylefixSingleSubheading
}

% @brief \resumeClusterItemList -- internal function used to withhold items
% @param 1: a list of \resumeItem (s)
\newcommand{\resumeClusterItemList}[1]{
    \resumeStyleItemListPaddingBefore
    \begin{itemize}[leftmargin=1.6em, rightmargin=0.007\textwidth]
        \justifying
        #1
    \end{itemize}
}

% @brief \resumeSkillCluster -- cluster of an aspect of skill
% @param aspect: whichever aspect to be enumerated, e.g. Languages, Tools
% @param enlist: list of skills, in plain text
\newkeycommand{\resumeSkillCluster}[
        aspect = Skill,
        enlist = List skills here,
]{
    \resumeStylefixSkillItem \resumeStyleItemPaddingBefore
    \small \item \textbf{\commandkey{aspect}}{: \commandkey{enlist}}
}

% @brief \resumeEducationCluster -- cluster describing education received
% @param institute: which university / high school / etc. did you attend
% @param role: which major / minor were you under pursuit
% @param beginTime: when did this education begin
% @param endTime: when did / will this education end (Present if ongoing)
% @param city: in which city did you attend this institute in
% @param region: in which country / region was the city located at
\newkeycommand{\resumeEducationCluster}[
        institute = University of Claydon,
        role = Undergraduate student in Computer Science,
        beginTime = Jan. 1970,
        endTime = Present,
        city = Claydon,
        region = Atlantica,
]{
    \resumeSubheadingTwoLinesInternal
        {\commandkey{institute}}
        {\commandkey{role}}
        {\commandkey{beginTime} -- \commandkey{endTime}}
        {\commandkey{city}, \commandkey{region}}
}

% @brief \resumeAwardCluster -- cluster describing award received
% @param prize: level of award, e.g. Silver medal / First prize / etc.
% @param name: name of award or canonical competition name
% @param time: time of competition taken place (or award received otherwise)
\newkeycommand{\resumeAwardCluster}[
        prize = Metal Medal,
        name = ICPC 1970 Claydon Regionals,
        time = Jan. 1970,
]{
    \resumeStylefixAwardItem \resumeStyleItemPaddingBefore \item
    % this part share the same code as \resumeSubheadingOneLineInternal does,
    % but does not use due to padding semantics' reasons.
    \begin{tabular*}{0.986\textwidth}{l@{\extracolsep{\fill}}r}
        \small \textit{\commandkey{prize}}, {\commandkey{name}} &
            \commandkey{time} \\
    \end{tabular*}
}

% @brief \resumeExperienceCluster -- cluster enlisting work experience
% @param role: work role during participation
% @param institute: company or other institute which you worked at
% @param beginTime: begin time of this work experience
% @param endTime: end time of this work experience (Present if ongoing)
% @param location: location of the institute (Remote if wfh)
% @param 1: list of cluster items
\newkeycommand{\resumeExperienceCluster}[
        role = Internship in Software Engineering,
        institute = Claydon Corporation,
        beginTime = Jan. 1970,
        endTime = Present,
        location = {Claydon, Atlantica},
][1]{
    \resumeSubheadingTwoLinesInternal
        {\commandkey{role}}
        {\commandkey{institute}}
        {\commandkey{beginTime} -- \commandkey{endTime}}
        {\commandkey{location}}
    \resumeClusterItemList{#1}
}

% @brief \resumeProjectCluster -- cluster describing project involved
% @param name: human-friendly name of project
% @param framework: languages, tech stacks or frameworks involved
% @param beginTime: begin time of development
% @param endTime: termination time of development
% @param 1: list of cluster items
\newkeycommand{\resumeProjectCluster}[
        name = Example Project,
        frameworks = Assembly,
        beginTime = Jan. 1970,
        endTime = Present,
][1]{
    \resumeSubheadingOneLineInternal
        {\textbf{\commandkey{name}} $|$ \emph{\commandkey{frameworks}}}
        {\commandkey{beginTime} -- \commandkey{endTime}}
    \resumeClusterItemList{#1}
}

% @brief \resumeItem -- cluster item, preceding plain text but do not require
%        any braces around the plain text subseding it
\newcommand{\resumeItem}{
    \resumeStyleItemPaddingBefore \item \small
}

% @brief \resumeWebLink -- lazily create a clickable hyperlink
% @param 1: link with https:// schema prepended
\newcommand{\resumeWebLink}[1]{
    \href{#1}{\underline{#1}}
}
